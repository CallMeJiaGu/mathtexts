\section{Введение}
\label{sec:Chapter1} \index{Chapter1}

В данной работе описывается модель \texttt{word2vecf} векторного представления слов на естественном языке. В отличие от популярной модели \texttt{word2vec}\footnote{\url{https://code.google.com/p/word2vec/}}, она основана на использовании деревьев зависимостей предложений. Согласно \cite{levy1}, она лучше справляется с задачей поиска синонимов слов. Тем не менее, соответствующий программный инструмент\footnote{\url{https://bitbucket.org/yoavgo/word2vecf/}} является плохо документированным и неудобным в использовании, а сама модель использовалась лишь в небольшом числе работ.

Целью работы является усовершенствование инструмента: переписывание вспомогательных программ для работы с моделью с языка программирования Python на язык C++, создание скрипта командной строки для запуска обучения модели, адаптация инструмента к формату входных данных, генерируемых как выходные данные синтаксическим анализатором (dependency parser) \texttt{RussianDependencyParser}\footnote
{\url{https://github.com/mathtexts/RussianDependencyParser/}} для русского языка.